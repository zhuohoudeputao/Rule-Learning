\documentclass[UTF8]{article}

\usepackage{ctex}
\usepackage{amsmath}


\title{规则学习 Rule Learning}
\author{}
\date{\today}

\begin{document}
\maketitle
\section{15.1 基本概念 Basic Concepts}
\subsection{规则 If-then}
    \begin{equation}
        \oplus \leftarrow f_1 \wedge f_2 \wedge \dots \wedge f_L
    \end{equation}
\subsection{覆盖 Cover}
    规则1覆盖西瓜数据集2.0中的样本1,2,3,4,5

    规则1:好瓜 $\leftarrow$ 根蒂=蜷缩 $\wedge$ 脐部=凹陷
    
    规则2:not好瓜 $\leftarrow$ 纹理=模糊

    我们将上面两条规则组成的规则集记为R
\subsubsection{覆盖可能造成的两个问题}
    1. 冲突:同一示例被不同规则覆盖,且判别结果不同。冲突消解:投票法、排序法、元规则法等  
        
    投票法:判别结果相同的规则数最多的为最终结果

    排序法:定义一种排序,最靠前的为最终结果

    元规则法:根据领域知识设定一些“规则的规则”,即指导规则的使用
        
    2. 不完全覆盖: 规则集不能完全覆盖整个数据集

    增加默认规则:“未被规则1,2覆盖的都不是好瓜”

\subsection{规则的表达能力}
    1. 命题规则:原子命题+逻辑连接词,例:R

    2. 一阶规则:原子公式+量词+逻辑连接词,例:
    \begin{equation}
        \forall X(N(\sigma (X)))\leftarrow N(X), ~where~ \sigma (X)=X+1
    \end{equation}
    
    3. 命题规则是一阶规则的特殊形式,一阶规则表达能力比命题规则要强

\section{15.2 序贯覆盖(以命题规则学习为例)}
\subsection{基本思想}
    逐条归纳,通过每次训练生成一条仅覆盖正例的规则,就去除掉那些覆盖的样例,用剩下的继续训练
\subsection{以西瓜数据集2.0的训练集为例的序贯覆盖}  
    1. 好瓜 $\leftarrow$ 色泽=青绿 
    ~覆盖了1、4、6,10、13、17,并不是全部都是正例  
    
    2. 好瓜 $\leftarrow$ 色泽=青绿 $\wedge$ 根蒂=蜷缩 
    ~覆盖了1、4,17,并不是全部都是正例  
    
    3. 好瓜 $\leftarrow$ 色泽=青绿 $\wedge$ 根蒂=蜷缩 $\wedge$ 敲声=浊响 
    ~覆盖了1,此时都是正例,就把这条规则加到规则集中,然后继续用剩下的样例进行训练
    
    缺陷:基于穷尽搜索,在属性和样例数量较多的时候可能会组合爆炸
\subsection{自顶向下与自底向上}
    自顶向下 top down:也就是上面这个过程,从一般到特殊,覆盖范围从大到小,其泛化性能更好(理由是更容易产生能较短的能覆盖更多正例的规则)
    
    自底向上 bottom up:从特殊到一般,覆盖范围从小到大,适用于一阶学习这种假设空间比较复杂的情况
\subsection{以西瓜数据集2.0的训练集为例的自顶向下}
    详见TopDown函数的输出

\section{15.3 剪枝优化}
\subsubsection{基本思想}
    通过剪枝前后发生的性能变化来判断是否进行剪枝,可以缓解过拟合的风险
    
    规则生成本质上是一个贪心搜索过程,
    也就是说,每一步都去寻找一个局部最优解,分阶段去逼近最优解。
    例如上面的分析过程,我们每次只在一些规则里面找到最好的规则
    (而不是去找全部的可能的规则进行比较),然后把它加进规则集里面,
    最后我们得到的规则集不一定是最好的解,但可能会是一个可行解。
    我们把这个局部最优解当成全局最优解来使用,可能会造成过拟合,
    于是通过剪枝来提高规则集的泛化性能。

    预剪枝是指生长过程中剪枝,后剪枝是指规则产生后剪枝。
\subsubsection{CN2算法}
    在预剪枝时,假设用规则集预测必须显著优于直接用训练集的后验概率分布(也就是训练集正反例数量的比率)进行猜测。使用了似然率统计量LRS来表示两种预测方法之间的差别。
    \begin{equation}
        LRS = 2(\hat{m}_+ \log_2{\frac{(\frac{\hat{m}_+}{\hat{m}_+ + \hat{m}_-})}{(\frac{m_+}{m_+ + m_-})}} + 
        \hat{m}_- \log_2{\frac{(\frac{\hat{m}_-}{\hat{m}_+ + \hat{m}_-})}{(\frac{m_-}{m_+ + m_-})}})
    \end{equation}
    当LRS越大的时候,说明规则集预测与直接用训练集分布进行猜测的差别越大。
    在数据量比较大的现实任务中,通常设置LRS很大(例如0.99)时才停止。
\subsubsection{REP}
    减错剪枝REP是后剪枝常用的策略,其基本做法是:
    将样例集划分为训练集T和验证集V,
    从T上学得规则集R后进行多轮剪枝,每一轮穷举可能的剪枝操作,
    包括删除规则中的某个或多个文字,删除整条规则等,
    然后用V对剪枝产生的规则集进行评估,保留最好的规则集进行下一轮剪枝,
    直到无法通过剪枝提高验证集上的性能为止。
\subsubsection{RIPPER算法}
    
\end{document}
